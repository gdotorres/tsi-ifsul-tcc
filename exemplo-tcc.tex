\documentclass[tcc,capa,table]{textsi}

\usepackage[utf8]{inputenc} % acentuacao
\usepackage[T1]{fontenc}

\usepackage{array,booktabs}
\usepackage{enumitem}
\usepackage{listings}
\usepackage{makecell}
\usepackage{graphicx,subfigure} % para inserir figuras
\usepackage{multirow}
\usepackage{xcolor}
\usepackage{pdfpages}
\usepackage{lipsum}

\title{Meu Trabalho}

\author{Torres}{Giovane de Oliveira}

\advisor[Prof.~Me.]{Torres}{Giovane de Oliveira}
\instituicaoadvisor{IFSul}

%\coadvisor[Prof.~Dr.]{Aguiar}{Marilton Sanchotene de}
%\collaborator[Prof.~Dr.]{Aguiar}{Marilton Sanchotene de}

\membroi{Prof.~Dr. Fulano de Souza e Silva}
\instituicaomembroi{UFI}
\membroii{Prof. Dr. Sicrano Silveira Rosa}
\instituicaomembroii{UFPG}
%\membroiii{Prof. Dr. Beltrano Pereira Tavares}
%\instituicaomembroiii{UFCR}

\keyword{Reuso de traços}
\keyword{Arquiteturas ARM}
\keyword{Desempenho}

\begin{document}

%\newcommand{\orientadora}{Orientadora}			%descomente caso tenhas orientadora
\newcommand{\coorientadora}{Coorientadora}		%descomente caso tenhas coorientadora

%\renewcommand{\advisorname}{\orientadora}       
%\renewcommand{\coadvisorname}{\coorientadora}	

\maketitle 

\sloppy

\fichacatalografica
\folhadeaprovacao

%Opcional - dedicatória
\begin{dedicatoria}
	Texto da Dedicatória
\end{dedicatoria}

%Opcional - agradecimentos
\begin{agradecimentos}
	Texto dos Agradecimentos
\end{agradecimentos}

%Opcional - epígrafe
\begin{epigrafe}
	"Texto da Epígrafe" \\
	Autor da Epígrafe
\end{epigrafe}

% Resumo em língua vernácula (a que você está escrevendo o documento)
\begin{abstract}
	Texto do resumo. Deve conter de 150 a 500 palavras.
\end{abstract}

% Resumo em língua estrangeira
\begin{englishabstract}%
 	{Palavras-chave em inglês}
 	Texto do resumo, em inglês.
\end{englishabstract}

%Lista de Figuras
\listoffigures

%Lista de Tabelas
\listoftables

%lista de abreviaturas e siglas
\begin{listofabbrv}{SN1} % <- Insira aqui a maior sigla da sua lista de siglas.
	% O conteúdo dentro dos colchetes é a sigla, e o texto ao lado é o seu significado
	\item[SN1] Sigla Número 1
\end{listofabbrv}

%Sumario
\tableofcontents

% Chapter é a seção primária do documento. Toda vez que é chamado este comando, uma quebra de página é adicionada;
\chapter{Seção Primária (\textit{chapter})}

\lipsum[1]

% Section é a seção secundária do documento. O restante é auto explicativo..
\section{Seção Secundária (\textit{section})}

\lipsum[1-2]

\subsection{Seção Terciária (\textit{subsection})}

\lipsum[1]

\subsubsection{Seção Quaternária (\textit{subsubsection})}

\lipsum[1]

\paragraph{Seção Quinária (\textit{paragraph})}

Exemplo de citação padrão~\cite{barber:92}. Uma outra citação~\cite{silva:18}. \lipsum[1-3]

Isso pode ser visto na Figura~\ref{fig:xitsu}.

\begin{figure}[!h]
\centering
\caption{Xitsu}
\label{fig:xitsu}
\includegraphics[scale=2]{xitsu.png} 
\source{Xitsu}
\end{figure}

\bibliography{exemplo-tcc}
\bibliographystyle{abnt}

%%%%%%%%%%%%%%%%%%%%%%%%%%%%%%%%%%%%%%%%%%%%%%%%%%%%%%%%%%%%

% NÃO MEXER - RELACIONADO À FORMATAÇÃO DE ANEXOS/APÊNDICES
\makeatletter%
\renewcommand*{\@seccntformat}[1]{\csname the#1\endcsname\hspace{0.2cm}}%
\makeatother%

%%%%%%%%%%%%%%%%%%%%%%%%%%%%%%%%%%%%%%%%%%%%%%%%%%%%%%%%%%%%%%

% Adiciona um apêndice
\append{Inverno}

% Adiciona um anexo
\annex{Verão}

\end{document}

